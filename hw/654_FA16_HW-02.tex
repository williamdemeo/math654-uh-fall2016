\documentclass[12pt]{amsart}

\usepackage[margin=2cm,nohead]{geometry}
\usepackage{multicol}
\usepackage{xfrac}
%
\usepackage{amsmath, amssymb, mathrsfs, amsthm}

\usepackage[all]{xy}
\usepackage{float}
\usepackage{wrapfig}


\pagestyle{empty}



\theoremstyle{definition}
\newtheorem{definition}{Definition}



\newtheorem{problem}{Problem}
\newtheorem{example}{Example}
\setcounter{problem}{7}
\setcounter{definition}{1}
\setcounter{example}{0}

\newcommand{\deriv}{\frac{d}{dx}}
\newcommand{\bbn}{\mathbb N}
\newcommand{\vect}[1]{\mathbf{#1}}
\newcommand{\ran}[1]{\mathrm{ran}(#1)}
\newcommand{\dom}[1]{\mathrm{dom}(#1)}
\newcommand{\vac} {\mathsf{Vac}}
\newcommand{\concat} {\hat{\ }}
\newcommand{\upto} {{\upharpoonright}}


\begin{document}

\begin{center}
\noindent{\large\textsc{Math 654 Homework \#2 (due on 9/16/2016)}}
\end{center}


\vspace{1em}

\noindent \textbf{Your name:}\\


\noindent\hrulefill

\vspace{-1.5em}

\noindent\hrulefill

\vspace{1em}




\begin{problem}
Prove that a set, $X$, is finite iff there is a relation such that it and its inverse order both well-order $X$.
\end{problem}





\begin{definition}
A set of strings over alphabet $A$ is a subset of $\bigcup_{n \in \omega} A^n$.  In other words, it is a set of functions whose domains are natural numbers and whose ranges are contained in $A$.  We denote by $\lambda$ the unique function with domain $0$.  The alphabet will not be mentioned if it is clear from context or does not need to be specified.  If $S$ is a set of strings and $x,y \in S$, we write $x \preceq y$ if $y \upto \dom{x} = x$.  If $\lnot ( x \preceq y \vee y \preceq x)$, then we write $x \perp y$ and say that $x$ and $y$ are incomparable; otherwise, we write $x \parallel y$ and say that $x$ and $y$ are comparable.  For a set of strings, $S$, $T[S] = \{ x : (\exists y\in S)\big( x \preceq y \big) \}$ is the prefix closure of $S$.
\end{definition}

\begin{definition}\label{*-1-def} Let $S$ and $P$ be two sets of strings. 
\begin{itemize}
\item $P * S = \{x y : x\in P \wedge y\in S\}$.
\item $P^{-1}S = \{ y : (\exists x\in P)\big( x y \in S \big)  \}$.
\end{itemize}
For notational simplicity, we define $x^{-1}S = \{x\}^{-1}S$, $P^{-1}x = P^{-1}\{x\}$, $x*S = \{x\}*S$ and $P*x = P*\{x\}$ for a string $x$.
\end{definition}

\begin{definition}\label{anti-def}
Given a set of strings, $S$, we call $P \subseteq T[S]$ a \emph{maximal antichain of $S$} if $(\forall x,y\in P)\big(x\perp y \vee x = y\big)$ and $(\forall x\in S)(\exists y\in P)(y \parallel x)$.  $P$ is a \emph{valid antichain of $S$} if $P$ is a maximal antichain of $S$ and $(\forall x,y \in  P)\big(x^{-1}T[S] = y^{-1}T[S]\big)$.  We define, $\vac(S) = \{P:P \mbox{ is a valid antichain of $S$}\}$.
\end{definition}


\begin{example}
Consider the following set of strings over the alphabet $\{a,b\}$:
$$S = \{ a^5,a^4b,a^2ba,a^2b^2,ba^4,ba^3b,baba,bab^2,b^2a^3,b^2a^2b,b^3a,b^4 \}.$$
%
Graphically, we can represent $S$ as a tree where branching left indicates an $a$ and branching right indicates a $b$.  In the picture below to the right, we highlight the four valid antichains of $S$: $P_0 = \{ \lambda \}$, $P_1 = \{ a^2,ba,b^2 \}$, $P_2 = \{ a^4,a^2b,ba^3,bab,b^2a^2,b^3 \}$ and $P_3 = S$.  Note that $S$ is only a valid antichain of itself because it contains no comparable strings.  The members of the four valid antichains are connected via dotted lines in the right picture ($P_0$ has only one member and therefore includes no dotted lines).  For reference a maximal antichain that is not valid is included in the picture on the left and its members are joined with a dotted line.
%
%
%
%
\begin{figure}[H]
\centering
\resizebox{12.5cm}{!}{
\xy
%
(-35,0)*++{\xy (0,0)*+[o]=<2pt>\cir<2pt>{}="0-0";
(16,-8)*+[o]=<2pt>\hbox{}*\frm{oo}="0-1";
(-16,-8)*+[o]=<2pt>\hbox{}*\frm{oo}="1-1";
(24,-16)*+[o]=<2pt>\hbox{}*\frm{oo}="0-2";
(8,-16)*+[o]=<2pt>\hbox{}*\frm{oo}="1-2";
(-24,-16)*+[o]=<2pt>\hbox{}*\frm{oo}="3-2";
(28,-24)*+[o]=<2pt>\hbox{}*\frm{oo}="0-3";
(20,-24)*+[o]=<2pt>\hbox{}*\frm{oo}="1-3";
(12,-24)*+[o]=<2pt>\hbox{}*\frm{oo}="2-3";
(4,-24)*+[o]=<2pt>\hbox{}*\frm{oo}="3-3";
(-20,-24)*+[o]=<2pt>\hbox{}*\frm{oo}="6-3";
(-28,-24)*+[o]=<2pt>\hbox{}*\frm{oo}="7-3";
(30,-32)*+[o]=<2pt>\hbox{\textbullet}*\frm{oo}="0-4";
(26,-32)*+[o]=<2pt>\hbox{\textbullet}*\frm{oo}="1-4";
(18,-32)*+[o]=<2pt>\hbox{}*\frm{oo}="3-4";
(14,-32)*+[o]=<2pt>\hbox{\textbullet}*\frm{oo}="4-4";
(10,-32)*+[o]=<2pt>\hbox{\textbullet}*\frm{oo}="5-4";
(2,-32)*+[o]=<2pt>\hbox{}*\frm{oo}="7-4";
(-18,-32)*+[o]=<2pt>\hbox{\textbullet}*\frm{oo}="12-4";
(-22,-32)*+[o]=<2pt>\hbox{\textbullet}*\frm{oo}="13-4";
(-30,-32)*+[o]=<2pt>\hbox{}*\frm{oo}="15-4";
(19,-40)*+[o]=<2pt>\hbox{\textbullet}*\frm{oo}="6-5";
(17,-40)*+[o]=<2pt>\hbox{\textbullet}*\frm{oo}="7-5";
(3,-40)*+[o]=<2pt>\hbox{\textbullet}*\frm{oo}="14-5";
(1,-40)*+[o]=<2pt>\hbox{\textbullet}*\frm{oo}="15-5";
(-29,-40)*+[o]=<2pt>\hbox{\textbullet}*\frm{oo}="30-5";
(-31,-40)*+[o]=<2pt>\hbox{\textbullet}*\frm{oo}="31-5";
"0-0";"0-1"**\dir{-};
"0-0";"1-1"**\dir{-};
"0-1";"0-2"**\dir{-};
"0-1";"1-2"**\dir{-};
"1-1";"3-2"**\dir{-};
"0-2";"0-3"**\dir{-};
"0-2";"1-3"**\dir{-};
"1-2";"2-3"**\dir{-};
"1-2";"3-3"**\dir{-};
"3-2";"6-3"**\dir{-};
"3-2";"7-3"**\dir{-};
"0-3";"0-4"**\dir{-};
"0-3";"1-4"**\dir{-};
"1-3";"3-4"**\dir{-};
"2-3";"4-4"**\dir{-};
"2-3";"5-4"**\dir{-};
"3-3";"7-4"**\dir{-};
"6-3";"12-4"**\dir{-};
"6-3";"13-4"**\dir{-};
"7-3";"15-4"**\dir{-};
"3-4";"6-5"**\dir{-};
"3-4";"7-5"**\dir{-};
"7-4";"14-5"**\dir{-};
"7-4";"15-5"**\dir{-};
"15-4";"30-5"**\dir{-};
"15-4";"31-5"**\dir{-};
%
% Not Valid Antichain
"0-3";"1-3"**\dir{.};
"1-3";"2-3"**\dir{.};
"2-3";"3-3"**\dir{.};
"3-3";"6-3"**\dir{.};
"6-3";"7-3"**\dir{.};
% Antichain 1
%"0-2";"1-2"**\dir{--};
%"1-2";"3-2"**\dir{--};
% Antichain 1
%"0-3";"3-4"**\dir{--};
%"3-4";"2-3"**\dir{--};
%"2-3";"7-4"**\dir{--};
%"7-4";"6-3"**\dir{--};
%"6-3";"15-4"**\dir{--};
\endxy }="G1";
%
%
(40,0)*++{\xy (0,0)*+[o]=<2pt>\cir<2pt>{}="0-0";
%
(4,0)*+[o]=<2pt>\hbox{$P_0$}="0-0-caption";
%
(16,-8)*+[o]=<2pt>\hbox{}*\frm{oo}="0-1";
(-16,-8)*+[o]=<2pt>\hbox{}*\frm{oo}="1-1";
(24,-16)*+[o]=<2pt>\hbox{}*\frm{oo}="0-2";
%
(28,-16)*+[o]=<2pt>\hbox{$P_1$}="0-2-caption";
%
(8,-16)*+[o]=<2pt>\hbox{}*\frm{oo}="1-2";
(-24,-16)*+[o]=<2pt>\hbox{}*\frm{oo}="3-2";
(28,-24)*+[o]=<2pt>\hbox{}*\frm{oo}="0-3";
%
(32,-24)*+[o]=<2pt>\hbox{$P_2$}="0-3-caption";
%
(20,-24)*+[o]=<2pt>\hbox{}*\frm{oo}="1-3";
(12,-24)*+[o]=<2pt>\hbox{}*\frm{oo}="2-3";
(4,-24)*+[o]=<2pt>\hbox{}*\frm{oo}="3-3";
(-20,-24)*+[o]=<2pt>\hbox{}*\frm{oo}="6-3";
(-28,-24)*+[o]=<2pt>\hbox{}*\frm{oo}="7-3";
(30,-32)*+[o]=<2pt>\hbox{\textbullet}*\frm{oo}="0-4";
%
(34,-32)*+[o]=<2pt>\hbox{$P_3$}="0-4-caption";
%
(26,-32)*+[o]=<2pt>\hbox{\textbullet}*\frm{oo}="1-4";
(18,-32)*+[o]=<2pt>\hbox{}*\frm{oo}="3-4";
(14,-32)*+[o]=<2pt>\hbox{\textbullet}*\frm{oo}="4-4";
(10,-32)*+[o]=<2pt>\hbox{\textbullet}*\frm{oo}="5-4";
(2,-32)*+[o]=<2pt>\hbox{}*\frm{oo}="7-4";
(-18,-32)*+[o]=<2pt>\hbox{\textbullet}*\frm{oo}="12-4";
(-22,-32)*+[o]=<2pt>\hbox{\textbullet}*\frm{oo}="13-4";
(-30,-32)*+[o]=<2pt>\hbox{}*\frm{oo}="15-4";
(19,-40)*+[o]=<2pt>\hbox{\textbullet}*\frm{oo}="6-5";
(17,-40)*+[o]=<2pt>\hbox{\textbullet}*\frm{oo}="7-5";
(3,-40)*+[o]=<2pt>\hbox{\textbullet}*\frm{oo}="14-5";
(1,-40)*+[o]=<2pt>\hbox{\textbullet}*\frm{oo}="15-5";
(-29,-40)*+[o]=<2pt>\hbox{\textbullet}*\frm{oo}="30-5";
(-31,-40)*+[o]=<2pt>\hbox{\textbullet}*\frm{oo}="31-5";
"0-0";"0-1"**\dir{-};
"0-0";"1-1"**\dir{-};
"0-1";"0-2"**\dir{-};
"0-1";"1-2"**\dir{-};
"1-1";"3-2"**\dir{-};
"0-2";"0-3"**\dir{-};
"0-2";"1-3"**\dir{-};
"1-2";"2-3"**\dir{-};
"1-2";"3-3"**\dir{-};
"3-2";"6-3"**\dir{-};
"3-2";"7-3"**\dir{-};
"0-3";"0-4"**\dir{-};
"0-3";"1-4"**\dir{-};
"1-3";"3-4"**\dir{-};
"2-3";"4-4"**\dir{-};
"2-3";"5-4"**\dir{-};
"3-3";"7-4"**\dir{-};
"6-3";"12-4"**\dir{-};
"6-3";"13-4"**\dir{-};
"7-3";"15-4"**\dir{-};
"3-4";"6-5"**\dir{-};
"3-4";"7-5"**\dir{-};
"7-4";"14-5"**\dir{-};
"7-4";"15-5"**\dir{-};
"15-4";"30-5"**\dir{-};
"15-4";"31-5"**\dir{-};
%
% Valid Antichain 1
"0-2";"1-2"**\dir{.};
"1-2";"3-2"**\dir{.};
% Valid Antichain 2
"0-3";"3-4"**\dir{.};
"3-4";"2-3"**\dir{.};
"2-3";"7-4"**\dir{.};
"7-4";"6-3"**\dir{.};
"6-3";"15-4"**\dir{.};
% Valid Antichain 3
"0-4";"1-4"**\dir{.};
"1-4";"6-5"**\dir{.};
"6-5";"7-5"**\dir{.};
"7-5";"4-4"**\dir{.};
"4-4";"5-4"**\dir{.};
"5-4";"14-5"**\dir{.};
"14-5";"15-5"**\dir{.};
"15-5";"12-4"**\dir{.};
"12-4";"13-4"**\dir{.};
"13-4";"30-5"**\dir{.};
"30-5";"31-5"**\dir{.};
%
\endxy }="G1";
%
%
%
%
\endxy
}
\caption{On the left, a maximal antichain that is not valid; on the right, all the valid antichains.}
\end{figure}
%
%  code for generating the picture:
%	2,2,14,4,2,4,6,4,0,5,1,5,2,5,3,5,8,5,9,5,10,5,11,5,24,5,25,5,26,5,27,5
%
In the next figure, we focus on the valid antichain $P_1$.
%
%
%
%
\begin{figure}[H]
\centering
\resizebox{6cm}{!}{
\xy
(0,0)*+[o]=<2pt>\cir<2pt>{}="0-0";
(16,-8)*+[o]=<2pt>\hbox{}*\frm{oo}="0-1";
(-16,-8)*+[o]=<2pt>\hbox{}*\frm{oo}="1-1";
(24,-16)*+[o]=<2pt>\hbox{}*\frm{oo}="0-2";
%
(28,-16)*+[o]=<2pt>\hbox{$P_1$}="0-2-caption";
(21,-18)*+[o]=<2pt>\hbox{$b^2$}="0-2-caption2";
%
(8,-16)*+[o]=<2pt>\hbox{}*\frm{oo}="1-2";
%
(4,-18)*+[o]=<2pt>\hbox{$ba$}="1-2-caption2";
%
(-24,-16)*+[o]=<2pt>\hbox{}*\frm{oo}="3-2";
%
(-27,-18)*+[o]=<2pt>\hbox{$a^2$}="2-2-caption2";
%
(28,-24)*+[o]=<2pt>\hbox{}*\frm{oo}="0-3";
(20,-24)*+[o]=<2pt>\hbox{}*\frm{oo}="1-3";
(12,-24)*+[o]=<2pt>\hbox{}*\frm{oo}="2-3";
(4,-24)*+[o]=<2pt>\hbox{}*\frm{oo}="3-3";
(-20,-24)*+[o]=<2pt>\hbox{}*\frm{oo}="6-3";
(-28,-24)*+[o]=<2pt>\hbox{}*\frm{oo}="7-3";
(30,-32)*+[o]=<2pt>\hbox{\textbullet}*\frm{oo}="0-4";
(26,-32)*+[o]=<2pt>\hbox{\textbullet}*\frm{oo}="1-4";
(18,-32)*+[o]=<2pt>\hbox{}*\frm{oo}="3-4";
(14,-32)*+[o]=<2pt>\hbox{\textbullet}*\frm{oo}="4-4";
(10,-32)*+[o]=<2pt>\hbox{\textbullet}*\frm{oo}="5-4";
(2,-32)*+[o]=<2pt>\hbox{}*\frm{oo}="7-4";
(-18,-32)*+[o]=<2pt>\hbox{\textbullet}*\frm{oo}="12-4";
(-22,-32)*+[o]=<2pt>\hbox{\textbullet}*\frm{oo}="13-4";
(-30,-32)*+[o]=<2pt>\hbox{}*\frm{oo}="15-4";
(19,-40)*+[o]=<2pt>\hbox{\textbullet}*\frm{oo}="6-5";
(17,-40)*+[o]=<2pt>\hbox{\textbullet}*\frm{oo}="7-5";
(3,-40)*+[o]=<2pt>\hbox{\textbullet}*\frm{oo}="14-5";
(1,-40)*+[o]=<2pt>\hbox{\textbullet}*\frm{oo}="15-5";
(-29,-40)*+[o]=<2pt>\hbox{\textbullet}*\frm{oo}="30-5";
(-31,-40)*+[o]=<2pt>\hbox{\textbullet}*\frm{oo}="31-5";
"0-0";"0-1"**\dir{.};
"0-0";"1-1"**\dir{.};
"0-1";"0-2"**\dir{.};
"0-1";"1-2"**\dir{.};
"1-1";"3-2"**\dir{.};
"0-2";"0-3"**\dir{-};
"0-2";"1-3"**\dir{-};
"1-2";"2-3"**\dir{-};
"1-2";"3-3"**\dir{-};
"3-2";"6-3"**\dir{-};
"3-2";"7-3"**\dir{-};
"0-3";"0-4"**\dir{-};
"0-3";"1-4"**\dir{-};
"1-3";"3-4"**\dir{-};
"2-3";"4-4"**\dir{-};
"2-3";"5-4"**\dir{-};
"3-3";"7-4"**\dir{-};
"6-3";"12-4"**\dir{-};
"6-3";"13-4"**\dir{-};
"7-3";"15-4"**\dir{-};
"3-4";"6-5"**\dir{-};
"3-4";"7-5"**\dir{-};
"7-4";"14-5"**\dir{-};
"7-4";"15-5"**\dir{-};
"15-4";"30-5"**\dir{-};
"15-4";"31-5"**\dir{-};
%
% Valid Antichain 1
"0-2";"1-2"**\dir{--};
"1-2";"3-2"**\dir{--};
%
%
%
%
%
\endxy
}
\caption{The identical subtrees below the elements of the valid antichain $P_1$.}
\end{figure}
%
%
%
%
Observe that the portions of the tree below each of $a^2$, $ba$ and $b^2$ are identical; the terminal nodes of all three sub-trees are $\{ a^3,a^2b,ab,b^2 \}$.  It is this equivalence of suffixes that makes $P_1$ a valid antichain.
\end{example}




\begin{problem}
Suppose that $P$ is a valid antichain of a set of strings $S$ and $Q$ is a valid antichain of $P$.  Prove that $Q$ is a valid antichain of~$S$.
\end{problem}




\end{document}